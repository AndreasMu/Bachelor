\documentclass[a4paper,10pt]{article}

\usepackage{ucs}
\usepackage[utf8x]{inputenc}
\usepackage{amsmath}
\usepackage{amssymb}
\usepackage{subfigure}
\usepackage[german]{babel}
\usepackage{fontenc}
\usepackage{graphicx}
\usepackage{capt-of}

\usepackage[dvips]{hyperref}

\date{03/15/15}

\begin{document}
 \section{Introduction}
 --Initial Problem (graph of obs data)\\
 --First thoughts (Hypothesis)\\
 --What Methods are viable?\\
 \section{Method}
 --Which Method did I use and why?\\
 --General approach (structure of main.py?)\\
 --What does Lumiradius.py do?\\
 For this approach I used two programs: main.py and lumiradius.py. \\
 
 --Detailed description of main.py[before lumiradius.py?]\\
 
 At the core of main.py lies a sequence of nested loops that spans a grid in distance-mass-age. The range of these three
 variables can be easily adjusted, as can the binning. I used a mass range from minmass=5$M_\odot$ to maxmass=50$M_\odot$, distance 
 from 0 to maxdistance=3kpc. maxage is a little more complicated: Because the main sequence age ($t_{ms}$) of a star is a strictly 
 monotonic increasing function of M the following is true: $maxage=t_{ms}(minmass)$. With a minmass of 5$M_\odot$ this translates 
 to maxage=104Myr\\
 The three informations I have about any given star, are its mass, its age and its distance from earth. From these three informations
 I need to derive its fractional main sequence age ($\tau$) and its apparent magnitude (V) and ultimately the probability density
 for all stars.\\ 
 Hurley Pols and Tout published a paper in 2000 in which they approximate the stellar evolution as a function of initial mass ($M_{ini}$),
 fractional main sequence age ($\tau$) and metalicity(Z). Now I need to know $\tau$. Using equation 5[citation needed] I know the 
 main sequence age 
 ($t_{ms}$) and $\tau$ then becomes: $\tau=\frac{t}{t_{ms}}$. Since I only include stars on the main sequence, I can safely include
 the condition: $\tau<1$ to cut down on computing time.
 Equation 12 and 13[citation needed] are very powerful equations to compute luminosity
 and radius for a star on the main sequence.
 \begin{equation}
  \log\frac{L_{MS}(t)}{L_{ZAMS}}=\alpha_L\tau+\beta_L\tau^\eta+\left(\log\frac{L_{TMS}}{L_{ZAMS}}-\alpha_L-\beta_L\right)\tau^2-\Delta L(\tau_1^2-\tau_2^2)
  \label{logL}
 \end{equation}
 \begin{equation}
  \log\frac{R_{MS}(t)}{R_{ZAMS}}=\alpha_R\tau+\beta_R\tau^{10}+\gamma\tau^{40}+\left(\log\frac{R_{TMS}}{R_{ZAMS}}-\alpha_R-\beta_R-\gamma\right)\tau^3-\Delta R(\tau_1^3-\tau_2^3)
  \label{logR}
 \end{equation}
 [insert dependancies of the variables in those equations]\\
 ...insert HRD illustrating lumiradius]\\
 
 The program does allow for freely changeable metalicities. For my purposes I use Z=0.02 for all stars to simulate
 a metalicity similar to that of our galactic neighborhood.\\
 I run this program for every possible mass-age tuple and save the results in a matrix This way I don't have to call the program for
 every distance-mass-age triple.\\
 With this I now know distance, mass, age, fractional main sequence age, luminosity and radius for any given star. I can now use these 
 informations to compute the apparent magnitudes.
 \begin{equation}
  M_{V}=V-5\cdot\log_{10}(distance)+5-Red
  \label{MV}
 \end{equation}
 \begin{equation}
  M_{bol}=M_{V}+BC
  \label{Mbol}
 \end{equation}
 \begin{equation}
  \frac{L}{L_\odot}=0.4\cdot(4.72-M_{bol})
  \label{LMbol}
 \end{equation}
 Where $M_{V}$ is the absolute visual magnitude, Red is the reddening as a function of distance, $M_{bol}$ is the absolute bolometric 
 magnitude and BC is the Bolometric Correction. Using equations \ref{MV}, \ref{Mbol} and \ref{LMbol} I can now compute the apparent visual
 magnitude V:
 \begin{equation}
  V=5\cdot\log_{10}(distance)-5+Red+4.72-\frac{L}{L_\odot\cdot0.4}-BC
 \end{equation}
 To get an approximation for reddening, I use figure 9 a graphic by AmoresLepine2005, I pick three points in the graph: 1:0.9, 2:2.25 and
 3:3.273]



 
 
 --optimization
 The binning can be freely adjusted. To optimize runtime of my program I conducted a few tests to adjust the binnings in all three dimensions.
 [graphics of 50-100-100, 100-50-100, 100-100-50 with caption: this is why I chose 1:2:2]\\
 
 I execute a pair of nested loops in mass and age to get all the data I need from lumiradius.py and save the results in two dimensional
 arrays. This way I won't have to call the function unnecessarily often in the core function.
 \section{results and conclusion}
 --
 \section{ending}
\end{document}
