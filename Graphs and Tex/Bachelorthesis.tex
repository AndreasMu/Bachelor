\documentclass[a4paper,10pt]{article}

\usepackage{ucs}
\usepackage[utf8x]{inputenc}
\usepackage{amsmath}
\usepackage{amssymb}
\usepackage{subfigure}
\usepackage[german]{babel}
\usepackage{fontenc}
\usepackage{graphicx}
\usepackage{capt-of}

\usepackage[dvips]{hyperref}

\date{03/15/15}

\begin{document}
 \section{Introduction}
 --Initial Problem (graph of obs data)\\
 --First thoughts (Hypothesis)\\
 --What Methods are viable?\\
 \section{Method}
 --Which Method did I use and why?\\
 --General approach (structure of main.py?)\\
 --What does Lumiradius.py do?\\
 \textbf{For this approach I used two programs: main.py and lumiradius.py.} \\
 
 At the core of my main program lies a sequence of nested loops that spans a grid in distance-mass-age. The range of these three
 variables can be easily adjusted, as can the binning. In accordance to the observational data I try to explain, distance goes from \
 0 to $r_{max}=3kpc$. I use a logarithmic binning for mass 
 from $M_{min}=5M_\odot$ to $M_{max}=50M_\odot$. The mass-distribution of my stars will follow the salpeter IMF. Because the IMF is
 naturally skewed towards smaller stars, I use a logarithmic mass-grid. Age spans from 0 to the maximum age the oldest star in my 
 simulation can reach. Because the main sequence age ($t_{ms}$) of a star 
 is a strictly monotonic increasing function of M the following is true: $t_{max}=t_{ms}(M_{min})$. With $M_{min}=5M_\odot$ this 
 translates to $t_{max}=104Myr$. This Axis will also be logarithmic. Even tho the IMF is skewed towards smaller, and thus long lived stars
 $t_{ms}$ drops faster, than M rises (By ~ a power of 2). This means, that there will still me much more stars on the lower 
 end of the age-scale, than on the higher end.\\
 The three informations I have about any given star, are its mass, its age and its distance from earth. From these three informations
 I need to derive its fractional main sequence age ($\tau$) and its apparent magnitude (V) and ultimately the probability density
 for all stars.\\ 
 Hurley Pols and Tout published a paper in 2000 in which they approximate the stellar evolution as a function of initial mass ($M_{ini}$),
 fractional main sequence age ($\tau$) and metalicity(Z). Now I need to know $\tau$. Using equation 5\textbf{citation needed} I know the 
 main sequence age 
 ($t_{ms}$) and $\tau$ then becomes: $\tau=\frac{t}{t_{ms}}$. Since I only include stars on the main sequence, I can safely include
 the condition: $\tau<1$ to cut down on computing time.
 Equation 12 and 13\textbf{citation needed} are very powerful equations to compute luminosity
 and radius for a star on the main sequence.
 \begin{equation}
  \log\frac{L_{MS}(t)}{L_{ZAMS}}=\alpha_L\tau+\beta_L\tau^\eta+\left(\log\frac{L_{TMS}}{L_{ZAMS}}-\alpha_L-\beta_L\right)\tau^2-\Delta L(\tau_1^2-\tau_2^2)
  \label{logL}
 \end{equation}
 \begin{equation}
  \log\frac{R_{MS}(t)}{R_{ZAMS}}=\alpha_R\tau+\beta_R\tau^{10}+\gamma\tau^{40}+\left(\log\frac{R_{TMS}}{R_{ZAMS}}-\alpha_R-\beta_R-\gamma\right)\tau^3-\Delta R(\tau_1^3-\tau_2^3)
  \label{logR}
 \end{equation}
 \textbf{detailed description of these values like $\alpha_L$?} \\
 \textbf{insert HRD illustrating lumiradius}\\
 
 The program does allow for freely changeable metalicities. For my purposes I use Z=0.02 for all stars to simulate
 a metalicity similar to that of our galactic neighborhood.\\
 I run this program for every possible mass-age tuple and save the results in a matrix This way I don't have to call the program for
 every distance-mass-age triple.\\
 With this I now know distance, mass, age, fractional main sequence age, luminosity and radius for any given star. I can now use these 
 informations to compute the apparent magnitudes.
 \begin{equation}
  M_{V}=V-5\cdot\log_{10}(distance)+5-Red
  \label{MV}
 \end{equation}
 \begin{equation}
  M_{bol}=M_{V}+BC
  \label{Mbol}
 \end{equation}
 \begin{equation}
  \frac{L}{L_\odot}=0.4\cdot(4.72-M_{bol})
  \label{LMbol}
 \end{equation}
 Where $M_{V}$ is the absolute visual magnitude, Red is the reddening as a function of distance, $M_{bol}$ is the absolute bolometric 
 magnitude and BC is the Bolometric Correction. Using equations \ref{MV}, \ref{Mbol} and \ref{LMbol} I can now compute the apparent visual
 magnitude V:
 \begin{equation}
  V=5\cdot\log_{10}(distance)-5+Red+4.72-\frac{L}{L_\odot\cdot0.4}-BC
 \end{equation}
 \textbf{To get an approximation for reddening, I use figure 9 a graphic by AmoresLepine2005, I pick three points in the graph: 1:0.9, 2:2.25 and
 3:3.273}\\
 \textbf{I need sources for these formulas.}\\
 The next thing I need to know are the probability densities for stars in distance $\left(\frac{dp}{dr}\right)$, mass 
 $\left(\frac{dp}{dm}\right)$ and age $\left(\frac{dp}{dt}\right)$. In my simulation I assume a homogenous distribution of stars. 
 This makes finding a probability density for distance very easy. \textbf{Since volume is a function of distance, I can also use }
 $\left(\frac{dp}{dV}\right)$:
 \begin{equation}
  \frac{dp}{dV}=\frac{1}{V_{tot}}=\frac{1}{\frac43\cdot\pi\cdot maxdistance^3}
  \label{dpdV}
 \end{equation}
 I assume a constant star formation rate, so the probability density in age would be similar to the density in distance. I do however use
 a logarithmic binning in age. Thus I can not simply use $\frac{dp}{dt}$ but instead need to find $\frac{dp}{d\log t}$ using the following:
  $\frac{dp}{dt}=\frac{1}{t_{ms}}\quad \land \quad d\log t=\frac{dt}{t \ln 10}$
 \begin{equation}
  \frac{dp}{d\log t}=\frac{t\ln 10}{t_{ms}}
 \end{equation}
 I assume, that the stars are distributed in mass following the salpeter initial mass function: $\frac{dp}{dM}=A\cdot M^{-2.35}$. Similar
 to the density function in age, I need to convert it for my logarithmic binning. 
 \begin{equation}
  \frac{dp}{d\log M}=\ln 10 \cdot A\cdot M^{-1.35}
 \end{equation}
 Where A is a normalization factor, that needs to be computed. 
 \begin{equation}
  1=\int_{M_{min}}^{M_{max}}A\cdot M^{-2.35} dM =\left[ -1.35\cdot A\cdot M^{-1.35}\right]_{M_{min}}^{M_{max}}
 \end{equation}
 \begin{equation}
  A= \frac{1.35}{M_{min}^{-1.35}-M_{max}^{-1.35}}
 \end{equation}
 With this information I can now formulate an overarching probability density function with regard to $\tau$
 \begin{equation}
  \frac{dp}{d\tau}=\frac{dp}{dV}dV \cdot \frac{dp}{d\log t}d\log t \cdot \frac{dp}{d\log M}d\log M\cdot \frac{1}{d\tau}
 \end{equation}
 

 
 \subsection{optimization}
 The binning can be freely adjusted. To optimize runtime of my program I conducted a few tests to adjust the binnings in all three dimensions.\\
 \textbf{graphics of 50-100-100, 100-50-100, 100-100-50 with caption: this is why I chose 1:2:2}\\
 
 I execute a pair of nested loops in mass and age to get all the data I need from lumiradius.py and save the results in two dimensional
 arrays. This way I won't have to call the function unnecessarily often in the core function.
 \section{results and conclusion}
 --
 \section{ending}
\end{document}
